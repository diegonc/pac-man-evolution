\documentclass{article}

\usepackage[spanish]{babel}
\usepackage[latin1]{inputenc}

\begin{document}
\section{Inversi�n de ciclos.}

El m�todo consiste en invertir los ciclos exteriores del m�todo b�sico para lograr que la matriz de resultado se recorra por columnas aprovechando mejor la memoria cache.

\section{Multiplicaci�n por columnas.}

Este m�todo realiza la multiplicaci�n de matrices en forma incremental, efectuando todas las operaciones posible con los elementos disponibles de una de las matrices y sumando los resultados parciales en la matriz resultado.

Utilizando las propiedades de las matrices se consiguen las siguientes equivalencias que proporcionanan fundamento matem�tico al m�todo. Por simplicidad, se utilizan matrices de 2x2.

$$
  \left[
	\begin{array}{cc}
		a_{11} & a_{12} \\
		a_{21} & a_{22}
	\end{array}
   \right]
   \left[
	\begin{array}{cc}
		b_{11} & b_{12} \\
		b_{21} & b_{22}
	\end{array}
   \right]
   =
   \left[
	\begin{array}{cc}
		c_{11} & c_{12} \\
		c_{21} & c_{22}
	\end{array}
   \right]
$$

El producto del miembro izquierdo se puede escribir como
$$
   \left(
	\left[
	\begin{array}{cc}
		a_{11} & 0 \\
		a_{21} & 0
	\end{array}
	\right]
        +
	\left[
	\begin{array}{cc}
		0 & a_{12} \\
		0 & a_{22}
	\end{array}
	\right]
   \right)
   \left(
   	\left[
	\begin{array}{cc}
		b_{11} & 0 \\
		b_{21} & 0
	\end{array}
	\right]
        +
	\left[
	\begin{array}{cc}
		0 & b_{12} \\
		0 & b_{22}
	\end{array}
	\right]
   \right)
$$

Y expandiendo los par�ntesis resulta

$$
	\left[
	\begin{array}{cc}
		a_{11} & 0 \\
		a_{21} & 0
	\end{array}
	\right]
	\left[
	\begin{array}{cc}
		b_{11} & 0 \\
		b_{21} & 0
	\end{array}
	\right]
        +
	\left[
	\begin{array}{cc}
		0 & a_{12} \\
		0 & a_{22}
	\end{array}
	\right]
	\left[
	\begin{array}{cc}
		b_{11} & 0 \\
		b_{21} & 0
	\end{array}
	\right]
	+
	\left[
	\begin{array}{cc}
		a_{11} & 0 \\
		a_{21} & 0
	\end{array}
	\right]
	\left[
	\begin{array}{cc}
		0 & b_{12} \\
		0 & b_{22}
	\end{array}
	\right]
        +
	\left[
	\begin{array}{cc}
		0 & a_{12} \\
		0 & a_{22}
	\end{array}
	\right]
	\left[
	\begin{array}{cc}
		0 & b_{12} \\
		0 & b_{22}
	\end{array}
	\right]
$$

As�, la matriz producto se puede descomponer en la siguiente suma
$$
	\left[
	\begin{array}{cc}
		a_{11} b_{11} & 0 \\
		a_{21} b_{11} & 0
	\end{array}
	\right]
        +
	\left[
	\begin{array}{cc}
		a_{12} b_{21} & 0 \\
		a_{22} b_{21} & 0
	\end{array}
	\right]
	+
	\left[
	\begin{array}{cc}
		0 & a_{11} b_{12} \\
		0 & a_{21} b_{12}
	\end{array}
	\right]
        +
	\left[
	\begin{array}{cc}
		0 & a_{12} b_{22} \\
		0 & a_{22} b_{22}
	\end{array}
	\right]
$$

Como se puede apreciar, las tres matrices se acceden por columnas. Esto resulta en una mejor utilizaci�n de la memoria cache.

Sin embargo, para funcione correctamente, requiere que la matriz donde se almacena el resultado haya sido inicializada, ya sea previamente en 0 o con los primeros valores calculados de cada elemento.
\end{document}
